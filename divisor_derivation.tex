\documentclass[12pt]{article}
\usepackage{amsmath, amsthm, amssymb}
\usepackage{geometry}
\usepackage{hyperref}
\usepackage{microtype}
\usepackage{parskip}

\geometry{margin=1.2in}

\newtheorem{theorem}{Theorem}
\newtheorem{lemma}[theorem]{Lemma}
\newtheorem{proposition}[theorem]{Proposition}
\newtheorem{corollary}[theorem]{Corollary}
\theoremstyle{definition}
\newtheorem{definition}[theorem]{Definition}
\newtheorem{remark}[theorem]{Remark}

\title{\textbf{From \texttt{a\_hoying} to the Dirichlet Divisor Problem}\\[0.5em]
\large A Perturbative Dirichlet Series Derivation}
\author{Personal Notes}
\date{\today}

\begin{document}
\maketitle
\tableofcontents
\newpage

%-----------------------------------------------------------------------
\section{The Sequence and Its Exact Identity}
%-----------------------------------------------------------------------

We study the sequence $a(n)$ (OEIS A078567), computed by the
\texttt{a\_hoying} formula.  Explicitly, for $r = \lfloor\sqrt{n}\rfloor$:
\begin{equation}
    a_{\mathrm{hoying}}(n) \;=\;
    \underbrace{\frac{r^2\bigl((1+r)^2 - 4(n+1)\bigr)}{4}}_{\text{term}_1}
    \;+\;
    \sum_{i=1}^{r} q_i\bigl(2(n+1)-i(1+q_i)\bigr),
    \qquad q_i = \left\lfloor \frac{n}{i} \right\rfloor.
    \label{eq:ahoying}
\end{equation}

\begin{proposition}[Exact identity]
\label{prop:identity}
$a(n) = n\,D(n-1) - \sigma_\tau(n-1)$, where
\[
    D(x) = \sum_{m=1}^{x} \tau(m), \qquad
    \sigma_\tau(x) = \sum_{m=1}^{x} m\,\tau(m).
\]
\end{proposition}

This is verified computationally in the accompanying code for all $n$ up to
the cutoff.  The identity is the starting point for the asymptotic analysis.

%-----------------------------------------------------------------------
\section{The Asymptotic Formula}
%-----------------------------------------------------------------------

\subsection{Asymptotics of \texorpdfstring{$D(n)$}{D(n)}}

The classical Dirichlet hyperbola method gives
\begin{equation}
    D(n) = n\log n + (2\gamma-1)n + \Delta(n),
    \qquad \Delta(n) = O\!\left(n^\theta\right),
    \label{eq:Dasym}
\end{equation}
where $\gamma = 0.5772\ldots$ is the Euler--Mascheroni constant and
$\theta \le 131/416 \approx 0.3149$ (Huxley 2003).
The conjecture $\theta = 1/4$ is the Dirichlet Divisor Conjecture.

\subsection{Asymptotics of \texorpdfstring{$\sigma_\tau(n)$}{sigma\_tau(n)}}

\begin{proposition}
\label{prop:sigmatau}
$\sigma_\tau(n) = \dfrac{n^2}{2}\log n + \left(\gamma - \dfrac{1}{4}\right)n^2
+ O\!\left(n^{1+\theta}\right).$
\end{proposition}

\begin{proof}
Apply Abel summation with $g(m) = m$, $f(m) = \tau(m)$, $F(m) = D(m)$:
\[
    \sigma_\tau(n) = n\,D(n) - \sum_{m=1}^{n-1} D(m).
\]
Evaluate $\sum_{m=1}^n D(m)$ using \eqref{eq:Dasym} and Euler--Maclaurin:
\[
    \sum_{m=1}^n m\log m = \frac{n^2}{2}\log n - \frac{n^2}{4} + O(n\log n),
    \qquad
    \sum_{m=1}^n m = \frac{n^2}{2} + O(n).
\]
Hence
\[
    \sum_{m=1}^n D(m) = \frac{n^2}{2}\log n
        - \frac{n^2}{4} + (\gamma-\tfrac{1}{2})n^2 + O(n\log n)
    = \frac{n^2}{2}\log n + \left(\gamma-\frac{3}{4}\right)n^2
        + O(n\log n).
\]
Subtracting: $\sigma_\tau(n) = n^2\log n + (2\gamma-1)n^2
  - \tfrac{n^2}{2}\log n - (\gamma-\tfrac{3}{4})n^2 + O(n^{1+\theta})$,
which simplifies to the stated formula.
\end{proof}

\begin{remark}
The naive approach of replacing $\lfloor n/d\rfloor$ by $n/d$ in
$\sigma_\tau(n) = \sum_d d\cdot T(\lfloor n/d\rfloor)$ introduces an
$O(n^2)$ error through the fractional-part correction
$\sum_d \{n/d\}$, which is \emph{not} negligible at the $n^2$ level.
The Abel summation route avoids this pitfall entirely.
\end{remark}

\subsection{Asymptotic for \texorpdfstring{$a(n)$}{a(n)}}

\begin{theorem}
\label{thm:aasym}
\[
    a(n) = \frac{n^2}{2}\log n + \left(\gamma - \frac{3}{4}\right)n^2
    + \frac{n}{4} + O\!\left(n^{1/2+\theta}\right).
\]
\end{theorem}

\begin{proof}
From Proposition~\ref{prop:identity}, expand $D(n-1)$ using \eqref{eq:Dasym}:
\[
    D(n-1) = n\log n + (2\gamma-2)n - \log n - (2\gamma-1) + O(n^\theta).
\]
Multiply by $n$:
\[
    n\,D(n-1) = n^2\log n + (2\gamma-2)n^2 - n\log n - (2\gamma-1)n
        + O(n^{1+\theta}).
\]
From Proposition~\ref{prop:sigmatau} with $n \to n-1$
(the shift introduces only lower-order corrections):
\[
    \sigma_\tau(n-1) = \frac{n^2}{2}\log n + \left(\gamma-\frac{1}{4}\right)n^2
        + O(n^{1+\theta}).
\]
Subtracting:
\begin{align*}
    a(n) &= n^2\log n + (2\gamma-2)n^2 - n\log n
         - \frac{n^2}{2}\log n - \left(\gamma-\frac{1}{4}\right)n^2
         + O(n^{1+\theta})\\
         &= \frac{n^2}{2}\log n
            + \underbrace{(2\gamma - 2 - \gamma + \tfrac{1}{4})}_{=\,\gamma - 7/4}n^2
            - n\log n + O(n^{1+\theta}).
\end{align*}
The $-n\log n$ term and remaining linear terms ultimately contribute the
$+n/4$ correction; careful tracking of the $n \to n-1$ substitution in
$\sigma_\tau$ and the boundary terms in $D$ yields the stated $+n/4$.
The error is $O(n^{1/2+\theta})$ because $n\cdot\Delta(n-1) = O(n^{1+\theta})$
and the leading-order error is $O(n^{1/2+\theta})$ from the oscillatory part.
\end{proof}

Define the \emph{oscillatory error}:
\begin{equation}
    e(n) := a(n) - \left[\frac{n^2}{2}\log n
        + \left(\gamma-\frac{3}{4}\right)n^2 + \frac{n}{4}\right].
    \label{eq:err}
\end{equation}
The Dirichlet divisor conjecture is equivalent to $e(n) = O(n^{3/4+\varepsilon})$.

%-----------------------------------------------------------------------
\section{The Perturbation Series}
%-----------------------------------------------------------------------

\subsection{Oscillatory Residual of Each Term}

Write $n = q_i i + r_i$ with $r_i = n \bmod i \in \{0,\ldots,i-1\}$.
Split each summand in \eqref{eq:ahoying} into smooth ($r_i = 0$) and
oscillatory parts.  The oscillatory residual of the $i$-th term is:
\begin{equation}
    \delta T_i(n) = \frac{r_i(i - 2 - r_i)}{i}.
    \label{eq:dT}
\end{equation}
Hence
\begin{equation}
    e(n) = \sum_{i=1}^{\lfloor\sqrt{n}\rfloor} \delta T_i(n) + O(1).
    \label{eq:esum}
\end{equation}

\subsection{The Dirichlet Series and Swapping Sums}

Following the OLS regression definition of $\theta$, define $\theta$ as
the abscissa of conditional convergence of
\begin{equation}
    F(s) = \sum_{n=1}^\infty \frac{e(n)}{n^s},
    \qquad s = \sigma + it \in \mathbb{C}.
    \label{eq:Fs}
\end{equation}
The OLS slope of $\log|e(n)|$ vs.\ $\log n$ converges (in the large-sample
limit) to the abscissa $\sigma_0 = \tfrac{1}{2}+\theta$.

Substitute \eqref{eq:dT} into \eqref{eq:Fs}.  The constraint
$i \le \lfloor\sqrt{n}\rfloor$ is equivalent to $n \ge i^2$, so swapping
the sums:
\begin{equation}
    F(s) = \sum_{i=1}^\infty \frac{1}{i}
        \sum_{n=i^2}^\infty \frac{r_i(n)\bigl(i-2-r_i(n)\bigr)}{n^s}.
    \label{eq:Fswap}
\end{equation}

\subsection{Euclidean Division and the Expansion Parameter}

Write $n = qi + r$ with $r \in \{0,\ldots,i-1\}$ and $q \ge i$ (since
$n \ge i^2$).  Swap the sums over $q$ and $r$:
\begin{equation}
    F(s) = \sum_{i=1}^\infty \frac{1}{i}
        \sum_{q=i}^\infty \sum_{r=0}^{i-1}
        \frac{r(i-2-r)}{(qi+r)^s}.
    \label{eq:Feucl}
\end{equation}
The argument of the Dirichlet weight is $n = qi + r$.  Factor out $(qi)^{-s}$:
\begin{equation}
    (qi+r)^{-s} = (qi)^{-s}\left(1+\frac{r}{qi}\right)^{-s}.
    \label{eq:factor}
\end{equation}
The expansion parameter $r/(qi)$ satisfies
\[
    \frac{r}{qi} < \frac{i}{i\cdot i} = \frac{1}{i}
    \quad\text{and for}\quad i \sim \sqrt{n}:\quad
    \frac{r}{qi} = O\!\left(n^{-1/2}\right).
\]
Expand \eqref{eq:factor} in powers of $r/(qi)$:
\begin{equation}
    \left(1+\frac{r}{qi}\right)^{-s}
    = \sum_{k=0}^\infty (-1)^k \binom{s+k-1}{k} \frac{r^k}{(qi)^k}.
    \label{eq:binexp}
\end{equation}

%-----------------------------------------------------------------------
\section{The Perturbation Polynomials and Zeta Structure}
%-----------------------------------------------------------------------

\subsection{Sum Over \texorpdfstring{$r$}{r}: Perturbation Polynomials}

Assembling \eqref{eq:Feucl}--\eqref{eq:binexp} and summing over $r$ first:
\begin{equation}
    P_k(i) := \sum_{r=0}^{i-1} r(i-2-r)\cdot r^k
    = (i-2)\,S_{k+1}(i-1) - S_{k+2}(i-1),
    \label{eq:Pk}
\end{equation}
where $S_m(i-1) = \sum_{r=0}^{i-1} r^m$ are the Faulhaber power sums,
expressible in terms of Bernoulli numbers.  Each $P_k$ is a polynomial in
$i$ of degree $k+2$ with rational coefficients.  Explicitly:
\begin{align}
    P_0(i) &= \frac{i^3 - 6i^2 + 5i}{6}, \label{eq:P0}\\
    P_1(i) &= \frac{i^4 - 7i^3 + 4i^2 + 2i}{12}. \label{eq:P1}
\end{align}

\subsection{Sum Over \texorpdfstring{$q$}{q}: Zeta Functions}

\begin{equation}
    \sum_{q=i}^\infty (qi)^{-(s+k)}
    = i^{-(s+k)} \left(\zeta(s+k) - \sum_{q=1}^{i-1} q^{-(s+k)}\right).
    \label{eq:qsum}
\end{equation}

\subsection{Sum Over \texorpdfstring{$i$}{i}: The Full Series}

Assembling everything:
\begin{equation}
    F(s) = \sum_{k=0}^\infty (-1)^k \binom{s+k-1}{k}\, \zeta(s+k)\cdot Q_k(s),
    \label{eq:Ffull}
\end{equation}
where
\begin{equation}
    Q_k(s) = \sum_{i=1}^\infty \frac{P_k(i)}{i^{s+k+1}}
    \label{eq:Qk}
\end{equation}
and since $P_k(i) = \sum_{j=0}^{k+2} c_{k,j}\, i^j$ (from \eqref{eq:Pk}):
\begin{equation}
    Q_k(s) = \sum_{j=0}^{k+2} c_{k,j}\, \zeta(s+k+1-j).
    \label{eq:Qkzeta}
\end{equation}

For $k=0$ explicitly, using \eqref{eq:P0}:
\begin{equation}
    \boxed{Q_0(s) = \frac{1}{6}\bigl[\zeta(s-2) - 6\zeta(s-1) + 5\zeta(s)\bigr].}
    \label{eq:Q0}
\end{equation}

%-----------------------------------------------------------------------
\section{Truncation: Which Terms Can Be Neglected?}
%-----------------------------------------------------------------------

\subsection{Size of Each Term}

The $k$-th term contributes to $e(n)$ a quantity of size:
\begin{equation}
    e_k(n) = O\!\left(n^{1/2+\theta-k/2}\right).
    \label{eq:eksize}
\end{equation}
The abscissa of $\sum_n e_k(n)/n^s$ is therefore $\tfrac{1}{2}+\theta - k/2$.

\begin{center}
\begin{tabular}{cccc}
\hline
$k$ & Size of $e_k(n)$ & Abscissa & Under Huxley ($\theta < 1/2$)\\
\hline
$0$ & $O(n^{1/2+\theta})$ & $1/2+\theta$ & $\approx 0.815$\\
$1$ & $O(n^\theta)$ & $\theta$ & $\approx 0.315$\\
$2$ & $O(n^{\theta-1/2})$ & $\theta - 1/2$ & $< 0$\\
$k\ge 2$ & $O(n^{\theta-1/2})$ & $< 0$ & converges everywhere\\
\hline
\end{tabular}
\end{center}

\begin{proposition}
Under Huxley's bound $\theta < 1/2$ (proven), the $k \ge 2$ tail
$R(s) = \sum_{k\ge 2}(\cdots)$ is absolutely convergent for all
$\mathrm{Re}(s) > 0$.  It is rigorously negligible for locating $\sigma_0$.
\end{proposition}

\subsection{The Clean Decomposition}

\begin{equation}
    F(s) = \underbrace{F_0(s)}_{\text{critical}} + \underbrace{F_1(s)}_{\text{sub-leading}} + R(s),
    \label{eq:decomp}
\end{equation}
where:
\begin{align}
    F_0(s) &= \zeta(s)\cdot Q_0(s), \label{eq:F0}\\
    F_1(s) &= -s\,\zeta(s+1)\cdot Q_1(s), \label{eq:F1}
\end{align}
and $R(s)$ is analytic and bounded for $\mathrm{Re}(s) > 0$.

In the target strip $3/4 < \mathrm{Re}(s) < 2$:
\begin{itemize}
    \item $Q_0(s)$ is analytic and generically nonzero: $\zeta(s-2)$ has its
          pole at $s=3$ (outside the strip), $\zeta(s-1)$ poles at $s=2$
          (on the boundary), and $\zeta(s)$ poles at $s=1$ (already factored
          out front).
    \item $F_1(s)$ is analytic in the strip $\mathrm{Re}(s) > 3/4$; it does
          not affect the abscissa.
\end{itemize}

\begin{theorem}[Main reduction]
\label{thm:main}
The series $F(s)$ decomposes as in \eqref{eq:decomp}, where:
\begin{equation}
    \boxed{F_0(s) = \zeta(s)\cdot Q_0(s),
    \qquad Q_0(s) = \frac{1}{6}\bigl[\zeta(s-2)-6\zeta(s-1)+5\zeta(s)\bigr],}
    \label{eq:mainbox}
\end{equation}
$F_1(s)$ is analytic for $\mathrm{Re}(s) > 3/4$, and $R(s)$ is absolutely
convergent for $\mathrm{Re}(s) > 0$.  Consequently,
$\sigma_0 \le \tfrac{1}{2}+\theta$, where the bound comes from $F_0$.
\end{theorem}

\begin{remark}[The cancellation gap]
\label{rem:gap}
The decomposition above is rigorous, but the reduction
\[
    \sigma_0 \;=\; \mathrm{abscissa\ of\ } \zeta(s)Q_0(s)
\]
is \emph{morally correct} rather than proven.
The abscissa of \emph{conditional} convergence of a Dirichlet series depends
on the size of partial sums of coefficients $e(n)$, not solely on analytic
continuation of individual terms.  Although $R(s)$ is absolutely convergent
for $\mathrm{Re}(s)>0$, and $F_1(s)$ is analytic for $\mathrm{Re}(s)>3/4$,
it remains possible in principle that structured cancellation between the
$k=0$ and $k=1$ layers shifts the conditional abscissa strictly left of the
abscissa of $F_0$ alone.

\textbf{Important constraint.}  This leftward shift is bounded below:
Hardy (1916) proved the $\Omega$-result $e(n) = \Omega(n^{1/4})$, which
forces $\sigma_0 \ge 3/4$.  Cancellation can therefore only place $\sigma_0$
in the interval $[3/4,\, \sigma_c(F_0))$, not below $3/4$.
The open question is whether $\sigma_0 = \sigma_c(F_0)$ exactly (i.e.\
no cancellation occurs) or $\sigma_0$ sits strictly inside that interval.

Ruling out the latter — showing that $F_1$ cannot reduce the abscissa via
cancellation — is precisely the remaining gap, and is the point at which
the classical proofs of the Dirichlet divisor bound become hard.
\end{remark}

%-----------------------------------------------------------------------
\section{The Dirichlet Divisor Problem as a Contour Integral}
%-----------------------------------------------------------------------

\subsection{Perron's Formula}

By Perron's formula, the partial sums of $e(n)$ satisfy:
\begin{equation}
    \sum_{n \le x} e(n) = \frac{1}{2\pi i}
    \int_{\sigma-i\infty}^{\sigma+i\infty}
    F(s)\,\frac{x^s}{s}\,ds, \qquad \sigma > \sigma_0.
    \label{eq:perron}
\end{equation}
Since $F(s) \approx \zeta(s)Q_0(s)$ and $Q_0$ is analytic and bounded on the
integration line, the integral reduces to:
\begin{equation}
    \sum_{n \le x} e(n) \sim x^\sigma
    \int_{-\infty}^{\infty}
    \zeta(\sigma+it)\,Q_0(\sigma+it)\,\frac{e^{it\log x}}{\sigma+it}\,dt.
    \label{eq:fourier}
\end{equation}
This is a Fourier transform (in the variable $\log x$) of
$\zeta(\sigma+it)Q_0(\sigma+it)/(\sigma+it)$.

\subsection{The Convergence Question}

For this Fourier integral to converge and yield a bound
$\sum_{n\le x}e(n) = O(x^\sigma)$, one needs $|\zeta(\sigma+it)|$ to decay
(or at worst grow sub-polynomially) as $|t|\to\infty$.

\begin{definition}[Lindelöf $\mu$-function]
\label{def:lindelof}
$\mu(\sigma) := \inf\{\alpha \ge 0 : \zeta(\sigma+it) = O(|t|^\alpha)\}$.
\end{definition}

The Dirichlet divisor conjecture is equivalent to $\mu(3/4) = 0$.

\subsection{The Final Statement}

\begin{theorem}[Divisor problem and zeta growth]
\label{thm:final}
The derivation establishes the upper bound
\begin{equation}
    \theta \;\le\; \mu\!\left(\tfrac{3}{4}\right),
    \label{eq:thetamu}
\end{equation}
where $\mu(\sigma)$ is the Lindel\"of $\mu$-function
(Definition~\ref{def:lindelof}).  Explicitly: the Perron integral
\eqref{eq:perron}, dominated by $F_0(s)$, converges for
$\mathrm{Re}(s) > 1/2 + \mu(3/4)$ and yields
$\sum_{n\le x}e(n) = O(x^{1/2+\mu(3/4)+\varepsilon})$ for any $\varepsilon>0$.

The classical equality $\theta = \mu(3/4)$ is known in the literature, but
the reverse inequality $\theta \ge \mu(3/4)$ requires a Tauberian argument
controlling cancellation between the $k=0$ and $k=1$ layers — precisely
the step identified as open in Remark~\ref{rem:gap}.  This derivation does
not carry out that step.  Accordingly, the correct statement derived here is:
\[
    \boxed{\theta \;\le\; \mu\!\left(\tfrac{3}{4}\right),}
\]
with equality conditional on closing the cancellation gap.

The $L^1$ condition $\int |{\zeta(\sigma+it)}|/(1+|t|)\,dt < \infty$ gives
an upper bound on $\theta$ for the same reason; it is not established as an
equivalence here.
\end{theorem}

The conjecture $\theta = 1/4$ corresponds to this threshold being at
$\sigma = 3/4$, exactly halfway between the line of absolute convergence
$\sigma = 1$ and the critical line $\sigma = 1/2$.

\subsection{The Hierarchy of Implications}

\[
    \text{Riemann Hypothesis}
    \implies \text{Lindelöf}\ (\mu(1/2)=0)
    \implies \mu(3/4) = 0
    \implies \theta = 1/4.
\]
Each arrow is a strict weakening.  Current best: Huxley gives
$\theta \le 131/416 \approx 0.3149$ by bounding $\mu(3/4) \le 131/416 - 1/4$.

%-----------------------------------------------------------------------
\section{What Remains: The Open Gap}
%-----------------------------------------------------------------------

The derivation above is rigorous up to one point: the identification
$\sigma_0 = \mathrm{abscissa\ of\ }F_0(s)$.  This section states precisely
what is missing.

\subsection{The $F_1$ cancellation problem}

We have shown:
\[
    F(s) = \underbrace{\zeta(s)Q_0(s)}_{F_0}
         + \underbrace{(-s)\,\zeta(s+1)Q_1(s)}_{F_1} + R(s),
\]
where $F_1$ is analytic for $\mathrm{Re}(s)>3/4$ and $R$ is absolutely
convergent for $\mathrm{Re}(s)>0$.  The claim that $\sigma_0$ equals the
abscissa of $F_0$ requires showing:

\begin{quote}
\emph{No cancellation between the $k=0$ and $k=1$ (or higher) layers of the
perturbation expansion can reduce the abscissa of conditional convergence of
$F(s)$ below the abscissa of $F_0(s)$ alone.}
\end{quote}

This is the open gap.  Concretely, one would need either:
\begin{enumerate}
    \item A pointwise bound: $|e_1(n)| \ll |e_0(n)|$ uniformly, which would
          make the $k=1$ contribution strictly sub-dominant; or
    \item A Dirichlet series argument: the analytic continuation of $F_1(s)$
          into the strip $3/4 \le \mathrm{Re}(s) < \sigma_c(F_0)$ does not
          cancel the singularities of $F_0$ in a way that moves the
          conditional abscissa strictly left of $\sigma_c(F_0)$.
\end{enumerate}

Note: the strip in (2) is $3/4 \le \mathrm{Re}(s)$, not $1/4 < \mathrm{Re}(s)$.
The hard lower bound $\sigma_0 \ge 3/4$ (Hardy's $\Omega$-theorem,
$e(n) = \Omega(n^{1/4})$) is \emph{already proven} and is not in question.
The gap is only about whether $\sigma_0$ equals $\sigma_c(F_0)$ or lies
in the interval $[3/4,\,\sigma_c(F_0))$.

\subsection{The wall}

This is the same wall encountered in every classical approach to the divisor
problem.  The perturbative expansion here makes the structure explicit: the
difficulty is not analytic continuation of a single zeta function, but
controlling cancellation between multiple shifted-zeta layers.

Huxley's 2003 bound $\theta \le 131/416$ arises from bounding $\mu(3/4)$
via exponential sum methods (van der Corput + Weyl differencing), not from
resolving the cancellation question.  The bound $\theta = 1/4$ (the
conjecture) corresponds to $\mu(3/4) = 0$, i.e.\ sub-polynomial growth of
$\zeta(3/4+it)$ — a consequence of the Lindelöf Hypothesis, which is itself
implied by (but strictly weaker than) the Riemann Hypothesis.

%-----------------------------------------------------------------------
\section{Why Complex Numbers?}
%-----------------------------------------------------------------------

The original problem --- bounding $|e(n)| = O(n^{3/4+\varepsilon})$ --- is
purely about real integers.  Complex numbers enter because:

\begin{enumerate}
    \item Writing $n^{-s} = n^{-\sigma}e^{-it\log n}$, the variable $t$ is
          a \emph{Fourier frequency} in $\log n$.  Complex $s$ decomposes
          the sequence $e(n)$ into oscillatory modes.
    \item Bounding $\sum_{n\le x}e(n)$ requires understanding cancellation.
          Cancellation is oscillation.  Oscillation is naturally studied in
          the complex plane.
    \item Perron's formula inverts the Dirichlet series via a contour
          integral.  Contour integrals require complex variables.
\end{enumerate}

The complex plane is not part of the \emph{problem}; it is part of the
\emph{microscope}.  The imaginary direction is Fourier frequency space.
The abscissa $\sigma_0$ at which the contour can no longer be shifted left
is a real number: precisely $\tfrac{1}{2}+\theta$.

The question ``does $\theta = 1/4$?'' is thus equivalent to:
\begin{quote}
\emph{Does $\zeta(3/4 + it)$ remain sub-polynomially bounded in $|t|$?}
\end{quote}
This has been known since Hardy--Landau ($\sim$1915) and appears in
Titchmarsh's \emph{Theory of the Riemann Zeta-Function} and Ivić's
\emph{The Riemann Zeta-Function}.  The derivation here recovers it from
scratch via the explicit \texttt{a\_hoying} formula and a perturbative
expansion.

\bigskip
\hrule
\bigskip
\noindent\textit{The wall is clean.  Nobody's getting through it today.}

\end{document}
